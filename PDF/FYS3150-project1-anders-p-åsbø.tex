\documentclass[english,notitlepage]{revtex4-1}  % defines the basic parameters of the document

% if you want a single-column, remove reprint

% allows special characters (including æøå)
\usepackage[utf8]{inputenc}
\usepackage[english]{babel}

%% note that you may need to download some of these packages manually, it depends on your setup.
%% I recommend downloading TeXMaker, because it includes a large library of the most common packages.

\usepackage{physics,amssymb}  % mathematical symbols (physics imports amsmath)
\usepackage{graphicx}         % include graphics such as plots
\usepackage{xcolor}           % set colors
\usepackage{hyperref}         % automagic cross-referencing (this is GODLIKE)
\usepackage{tikz}             % draw figures manually
\usepackage{listings}         % display code
\usepackage{subfigure}        % imports a lot of cool and useful figure commands
\usepackage{lipsum}

\usepackage{lmodern}

% (2) specify encoding
\usepackage[T1]{fontenc}
\usepackage{textcomp}
%\usepackage{unicode-math}
\usepackage{float}
\usepackage{balance}

% defines the color of hyperref objects
% Blending two colors:  blue!80!black  =  80% blue and 20% black
\hypersetup{ % this is just my personal choice, feel free to change things
    colorlinks,
    linkcolor={red!50!black},
    citecolor={blue!50!black},
    urlcolor={blue!80!black}}

%% Defines the style of the programming listing
%% This is actually my personal template, go ahead and change stuff if you want
\lstset{ %
	inputpath=,
	backgroundcolor=\color{white!88!black},
	basicstyle={\ttfamily\scriptsize},
	commentstyle=\color{magenta},
	language=Python,
	morekeywords={True,False},
	tabsize=4,
	stringstyle=\color{green!55!black},
	frame=single,
	keywordstyle=\color{blue},
	showstringspaces=false,
	columns=fullflexible,
	keepspaces=true}

\lstset{literate=
  {á}{{\'a}}1 {é}{{\'e}}1 {í}{{\'i}}1 {ó}{{\'o}}1 {ú}{{\'u}}1
  {Á}{{\'A}}1 {É}{{\'E}}1 {Í}{{\'I}}1 {Ó}{{\'O}}1 {Ú}{{\'U}}1
  {à}{{\`a}}1 {è}{{\`e}}1 {ì}{{\`i}}1 {ò}{{\`o}}1 {ù}{{\`u}}1
  {À}{{\`A}}1 {È}{{\'E}}1 {Ì}{{\`I}}1 {Ò}{{\`O}}1 {Ù}{{\`U}}1
  {ä}{{\"a}}1 {ë}{{\"e}}1 {ï}{{\"i}}1 {ö}{{\"o}}1 {ü}{{\"u}}1
  {Ä}{{\"A}}1 {Ë}{{\"E}}1 {Ï}{{\"I}}1 {Ö}{{\"O}}1 {Ü}{{\"U}}1
  {â}{{\^a}}1 {ê}{{\^e}}1 {î}{{\^i}}1 {ô}{{\^o}}1 {û}{{\^u}}1
  {Â}{{\^A}}1 {Ê}{{\^E}}1 {Î}{{\^I}}1 {Ô}{{\^O}}1 {Û}{{\^U}}1
  {œ}{{\oe}}1 {Œ}{{\OE}}1 {æ}{{\ae}}1 {Æ}{{\AE}}1 {ß}{{\ss}}1
  {ű}{{\H{u}}}1 {Ű}{{\H{U}}}1 {ő}{{\H{o}}}1 {Ő}{{\H{O}}}1
  {ç}{{\c c}}1 {Ç}{{\c C}}1 {ø}{{\o}}1 {å}{{\r a}}1 {Å}{{\r A}}1
  {€}{{\euro}}1 {£}{{\pounds}}1 {«}{{\guillemotleft}}1
  {»}{{\guillemotright}}1 {ñ}{{\~n}}1 {Ñ}{{\~N}}1 {¿}{{?`}}1
}

%% USEFUL LINKS:
%%
%%   UiO LaTeX guides:        https://www.mn.uio.no/ifi/tjenester/it/hjelp/latex/
%%   mathematics:             https://en.wikibooks.org/wiki/LaTeX/Mathematics

%%   PHYSICS !                https://mirror.hmc.edu/ctan/macros/latex/contrib/physics/physics.pdf

%%   the basics of Tikz:       https://en.wikibooks.org/wiki/LaTeX/PGF/TikZ
%%   all the colors!:          https://en.wikibooks.org/wiki/LaTeX/Colors
%%   how to draw tables:       https://en.wikibooks.org/wiki/LaTeX/Tables
%%   code listing styles:      https://en.wikibooks.org/wiki/LaTeX/Source_Code_Listings
%%   \includegraphics          https://en.wikibooks.org/wiki/LaTeX/Importing_Graphics
%%   learn more about figures  https://en.wikibooks.org/wiki/LaTeX/Floats,_Figures_and_Captions
%%   automagic bibliography:   https://en.wikibooks.org/wiki/LaTeX/Bibliography_Management  (this one is kinda difficult the first time)
%%   REVTeX Guide:             http://www.physics.csbsju.edu/370/papers/Journal_Style_Manuals/auguide4-1.pdf
%%
%%   (this document is of class "revtex4-1", the REVTeX Guide explains how the class works)


%% CREATING THE .pdf FILE USING LINUX IN THE TERMINAL
%%
%% [terminal]$ pdflatex template.tex
%%
%% Run the command twice, always.
%% If you want to use \footnote, you need to run these commands (IN THIS SPECIFIC ORDER)
%%
%% [terminal]$ pdflatex template.tex
%% [terminal]$ bibtex template
%% [terminal]$ pdflatex template.tex
%% [terminal]$ pdflatex template.tex
%%
%% Don't ask me why, I don't know.

\usepackage{thmtools}
\DeclareMathOperator{\nullspace}{Nul}
\DeclareMathOperator{\collspace}{Col}
\DeclareMathOperator{\rref}{Rref}
%%\DeclareMathOperator{\dim}{Dim}

 % "meq": must be equal
\newcommand{\meq}{\overset{!}{=}}

\newcommand{\R}{\mathbb{R}}
\newcommand*\Heq{\ensuremath{\overset{\kern2pt L'H}{=}}}
\usepackage{bm}
\newcommand{\uveci}{{\bm{\hat{\textnormal{\bfseries\i}}}}}
\newcommand{\uvecj}{{\bm{\hat{\textnormal{\bfseries\j}}}}}
\DeclareRobustCommand{\uvec}[1]{{%
  \ifcsname uvec#1\endcsname
     \csname uvec#1\endcsname
   \else
    \bm{\hat{\mathbf{#1}}}%
   \fi
}}
\usepackage{siunitx}

\makeatletter
\newcommand*{\balancecolsandclearpage}{%
  \close@column@grid
  \cleardoublepage
  \twocolumngrid
}
\makeatother

\newcounter{subproject}
\renewcommand{\thesubproject}{\alph{subproject}}
\newenvironment{subproj}{
\begin{description}
	\item[\refstepcounter{subproject}(\thesubproject)]
}{\end{description}}


\begin{document}
\title{Numerical integration using linear algebra}   % self-explanatory
\author{Anders P. Åsbø}               % self-explanatory
\date{\today}
\noaffiliation                            % ignore this

\begin{abstract}
derp
\end{abstract}

\maketitle
\tableofcontents

\section{Introduction}\label{sec:1}

\section{Formalism}\label{sec:2}

\section{Implementation}\label{sec:3}

\section{Analysis}\label{sec:4}

\section{Conclusion}\label{sec:5}

\appendix
\section{Program files} \label{A:1}
All code for this report was written in Python 3.6, and the complete set of files can be found at \url{https://github.com/FunkMarvel/CompPhys-Project-1}.
\subsection{project.py}\label{A:11}
\begin{lstlisting}
# Project 1 FYS3150, Anders P. Åsbø
# general tridiagonal matrix.
import data_generator as gen
import numpy as np
import os
import matplotlib.pyplot as plt
import timeit as time


def main():
    """Program solves matrix equation Au=f, using decomposition, forward
    substitution and backward substitution, for a tridiagonal, NxN matrix A."""
    init_data()  # initialising data

    # performing decomp. and forward and backward sub.:
    decomp_and_forward_and_backward_sub()

    save_sol()  # saving numerical solution in "data_files" directory.

    plot_solutions()  # plotting numerical solution vs analytical solution.

    plt.show()  # displaying plot.


def init_data():
    """Initialising data for program as global variables."""
    global dir, N, name, x, h, anal_sol, u, d, d_prime, a, b, g, g_prime
    dir = os.path.dirname(os.path.realpath(__file__))  # current directory.

    # defining number of rows and columns in matrix:
    N = int(eval(input("Specify number of data points N: ")))
    # defining common label for data files:
    name = input("Label of data-sets without file extension: ")

    x = np.linspace(0, 1, N)  # array of normalized positions.
    h = (x[0]-x[-1])/N  # defining step-siz.

    gen.generate_data(x, name)  # generating dataanal_name set.
    anal_sol = np.loadtxt("%s/data_files/anal_solution_for_%s.dat" %
                          (dir, name))

    u = np.empty(N)  # array for unkown values.
    d = np.full(N, 2)  # array for diagonal elements.
    d_prime = np.empty(N)  # array for diagonal after decom. and sub.
    a = np.full(N-1, -1)  # array for upper, off-center diagonal.
    b = np.full(N-1, -1)  # array for lower, off-center diagonal.
    # array for g in matrix eq. Au=g.
    f = np.loadtxt("%s/data_files/%s.dat" % (dir, name))
    g = f*h**2
    g_prime = np.empty(N)  # array for g after decomp. and sub.


def decomp_and_forward_and_backward_sub():
    """Function that performs the matrix decomposition and forward
    and backward substitution."""
    # setting boundary conditions:
    u[0], u[-1] = 0, 0
    d_prime[0] = d[0]
    g_prime[0] = g[0]

    start = time.default_timer()
    for i in range(1, len(u)):  # performing decomp. and forward sub.
        decomp_factor = b[i-1]/d_prime[i-1]
        d_prime[i] = d[i] - a[i-1]*decomp_factor
        g_prime[i] = g[i] - g_prime[i-1]*decomp_factor

    for i in reversed(range(1, len(u)-1)):  # performing backward sub.
        u[i] = (g_prime[i]-a[i]*u[i+1])/d_prime[i]
    end = time.default_timer()
    print("Time spent on loop %e" % (end-start))


def save_sol():
    """Function for saving numerical solution in data_files directory
    with prefix "solution"."""
    path = "%s/data_files/solution_%s.dat" % (dir, name)
    np.savetxt(path, u, fmt="%f")


def plot_solutions():
    """Function for plotting numerical vs analytical solutions."""
    x_prime = np.linspace(x[0], x[-1], len(anal_sol))

    plt.figure()
    plt.plot(x, u, label="Numerical solve")
    plt.plot(x_prime, anal_sol, label="Analytical solve")
    plt.title("Integrating with a %iX%i tridiagonal matrix" % (N, N))
    plt.xlabel(r"$x \in [0,1]$")
    plt.ylabel(r"$u(x)$")
    plt.legend()
    plt.grid()


if __name__ == '__main__':
    main()


# example run:
"""
$ python3 project.py
Specify number of data points N: 1000
Label of data-sets without file extension: num1000x1000
"""
# a plot is displayed, and the data is saved to the data_files directory.
\end{lstlisting}

\subsection{project\_specialized.py}\label{A:12}
\begin{lstlisting}
# Project 1 FYS3150, Anders P. Åsbø
import data_generator as gen
import numpy as np
import os
import matplotlib.pyplot as plt
import timeit as time


def main():
    """Program solves matrix equation Au=f, using decomposition, forward
    substitution and backward substitution, for a Toeplitz, NxN matrix A."""
    init_data()  # initialising data

    # performing decomp. and forward and backward sub.:
    decomp_and_forward_and_backward_sub()

    save_sol()  # saving numerical solution in "data_files" directory.

    plot_solutions()  # plotting numerical solution vs analytical solution.

    plt.show()  # displaying plot.


def init_data():
    """Initialising data for program as global variables."""
    global dir, N, name, x, h, anal_sol, u, d, d_prime, a, b, g, g_prime
    dir = os.path.dirname(os.path.realpath(__file__))  # current directory.

    # defining number of rows and columns in matrix:
    N = int(eval(input("Specify number of data points N: ")))
    # defining common label for data files:
    name = input("Label of data-sets without file extension: ")

    x = np.linspace(0, 1, N)  # array of normalized positions.
    h = (x[0]-x[-1])/N  # defining step-siz.

    gen.generate_data(x, name)  # generating dataanal_name set.
    anal_sol = np.loadtxt("%s/data_files/anal_solution_for_%s.dat" %
                          (dir, name))

    u = np.empty(N)  # array for unkown values.
    s = np.arange(1, N+1)
    d_prime = 2*(s)/(2*(s+1))  # pre-calculating the 1/d_prime factors.
    f = np.loadtxt("%s/data_files/%s.dat" % (dir, name))
    g = f*h**2
    g_prime = np.empty(N)  # array for g after decomp. and sub.


def decomp_and_forward_and_backward_sub():
    """Function that performs the matrix decomposition and forward
    and backward substitution."""
    # setting boundary conditions:
    u[0], u[-1] = 0, 0
    g_prime[0] = g[0]
    start = time.default_timer()
    for i in range(1, len(u)):  # performing decomp. and forward sub.
        g_prime[i] = g[i] + g_prime[i-1]*d_prime[i-1]

    for i in reversed(range(1, len(u)-1)):  # performing backward sub.
        u[i] = (g_prime[i] + u[i+1])*d_prime[i-1]

    end = time.default_timer()
    print("Time spent on loop %e" % (end-start))


def save_sol():
    """Function for saving numerical solution in data_files directory
    with prefix "solution"."""
    path = "%s/data_files/solution_%s.dat" % (dir, name)
    np.savetxt(path, u, fmt="%f")


def plot_solutions():
    """Function for plotting numerical vs analytical solutions."""
    x_prime = np.linspace(x[0], x[-1], len(anal_sol))

    plt.figure()
    plt.plot(x, u, label="Numerical solve")
    plt.plot(x_prime, anal_sol, label="Analytical solve")
    plt.title("Integrating with a %iX%i tridiagonal matrix" % (N, N))
    plt.xlabel(r"$x \in [0,1]$")
    plt.ylabel(r"$u(x)$")
    plt.legend()
    plt.grid()


if __name__ == '__main__':
    main()


# example run:
"""
$ python3 project_specialized.py
Specify number of data points N: 1000
Label of data-sets without file extension: opti1000x1000
"""
# a plot is displayed, and the data is saved to the data_files directory.

\end{lstlisting}

\subsection{data\_generator.py}
\begin{lstlisting}
# create data set for numerical testing
import numpy as np
import os

dir = os.path.dirname(os.path.realpath(__file__))


def main():
    test_generate_data()
    test_generate_tridiagonal()


def generate_data(x, name):
    """Function that generates a set of u''(x) data, as well as the
    corresponding, analytical u(x). The data is saved to text"""
    data = 100*np.exp(-10*x)
    path = "%s/data_files/%s.dat" % (dir, name)
    np.savetxt(path, data, fmt="%f")

    x_prime = np.linspace(x[0], x[-1], 1000)
    analytical_solution = 1-(1-np.exp(-10))*x_prime-np.exp(-10*x_prime)
    analytical_solution[0], analytical_solution[-1] = 0, 0
    anal_name = "%s/data_files/anal_solution_for_%s.dat" % (dir, name)
    np.savetxt(anal_name, analytical_solution, fmt="%f")


def generate_tridiagonal(N):
    """Function that generates a Nx3 array with each column corresponding to
    the non-zero elements in a tridiagonal matrix.
    "b" (mat_data[:,0]) is the lower diagonal,
    "d" (mat_data[:,1]) is the diagonal,
    and "a" (mat_data[:,2]) is the upper diagonal."""
    mat_data = np.random.randint(1, 100, size=(N, 3))
    np.savetxt("b-d-a_tridiagonal.dat", mat_data, fmt="%f")


def test_generate_data():
    """Generates test data if run as stand-alone program."""
    x = np.linspace(0, 1, 1001)
    test_name = "Test_data"
    generate_data(x, test_name)


def test_generate_tridiagonal():
    """Generates test data for tridiagonal."""
    generate_tridiagonal(100)
    print(np.loadtxt("b-d-a_tridiagonal.dat"))


if __name__ == '__main__':
    main()

\end{lstlisting}

\bibliography{kilder}{}


\end{document}
