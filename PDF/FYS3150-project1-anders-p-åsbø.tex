\documentclass[english,notitlepage]{revtex4-1}  % defines the basic parameters of the document

% if you want a single-column, remove reprint

% allows special characters (including æøå)
\usepackage[utf8]{inputenc}
\usepackage[english]{babel}

%% note that you may need to download some of these packages manually, it depends on your setup.
%% I recommend downloading TeXMaker, because it includes a large library of the most common packages.

\usepackage{physics,amssymb}  % mathematical symbols (physics imports amsmath)
\usepackage{graphicx}         % include graphics such as plots
\usepackage{xcolor}           % set colors
\usepackage{hyperref}         % automagic cross-referencing (this is GODLIKE)
\usepackage{tikz}             % draw figures manually
\usepackage{listings}         % display code
\usepackage{subfigure}        % imports a lot of cool and useful figure commands
\usepackage{lipsum}

\usepackage{lmodern}

% (2) specify encoding
\usepackage[T1]{fontenc}
\usepackage{textcomp}
%\usepackage{unicode-math}
\usepackage{float}
\usepackage{balance}

% defines the color of hyperref objects
% Blending two colors:  blue!80!black  =  80% blue and 20% black
\hypersetup{ % this is just my personal choice, feel free to change things
    colorlinks,
    linkcolor={red!50!black},
    citecolor={blue!50!black},
    urlcolor={blue!80!black}}

%% Defines the style of the programming listing
%% This is actually my personal template, go ahead and change stuff if you want
\lstset{ %
	inputpath=,
	backgroundcolor=\color{white!88!black},
	basicstyle={\ttfamily\scriptsize},
	commentstyle=\color{magenta},
	language=Python,
	morekeywords={True,False},
	tabsize=4,
	stringstyle=\color{green!55!black},
	frame=single,
	keywordstyle=\color{blue},
	showstringspaces=false,
	columns=fullflexible,
	keepspaces=true}

\lstset{literate=
  {á}{{\'a}}1 {é}{{\'e}}1 {í}{{\'i}}1 {ó}{{\'o}}1 {ú}{{\'u}}1
  {Á}{{\'A}}1 {É}{{\'E}}1 {Í}{{\'I}}1 {Ó}{{\'O}}1 {Ú}{{\'U}}1
  {à}{{\`a}}1 {è}{{\`e}}1 {ì}{{\`i}}1 {ò}{{\`o}}1 {ù}{{\`u}}1
  {À}{{\`A}}1 {È}{{\'E}}1 {Ì}{{\`I}}1 {Ò}{{\`O}}1 {Ù}{{\`U}}1
  {ä}{{\"a}}1 {ë}{{\"e}}1 {ï}{{\"i}}1 {ö}{{\"o}}1 {ü}{{\"u}}1
  {Ä}{{\"A}}1 {Ë}{{\"E}}1 {Ï}{{\"I}}1 {Ö}{{\"O}}1 {Ü}{{\"U}}1
  {â}{{\^a}}1 {ê}{{\^e}}1 {î}{{\^i}}1 {ô}{{\^o}}1 {û}{{\^u}}1
  {Â}{{\^A}}1 {Ê}{{\^E}}1 {Î}{{\^I}}1 {Ô}{{\^O}}1 {Û}{{\^U}}1
  {œ}{{\oe}}1 {Œ}{{\OE}}1 {æ}{{\ae}}1 {Æ}{{\AE}}1 {ß}{{\ss}}1
  {ű}{{\H{u}}}1 {Ű}{{\H{U}}}1 {ő}{{\H{o}}}1 {Ő}{{\H{O}}}1
  {ç}{{\c c}}1 {Ç}{{\c C}}1 {ø}{{\o}}1 {å}{{\r a}}1 {Å}{{\r A}}1
  {€}{{\euro}}1 {£}{{\pounds}}1 {«}{{\guillemotleft}}1
  {»}{{\guillemotright}}1 {ñ}{{\~n}}1 {Ñ}{{\~N}}1 {¿}{{?`}}1
}

%% USEFUL LINKS:
%%
%%   UiO LaTeX guides:        https://www.mn.uio.no/ifi/tjenester/it/hjelp/latex/
%%   mathematics:             https://en.wikibooks.org/wiki/LaTeX/Mathematics

%%   PHYSICS !                https://mirror.hmc.edu/ctan/macros/latex/contrib/physics/physics.pdf

%%   the basics of Tikz:       https://en.wikibooks.org/wiki/LaTeX/PGF/TikZ
%%   all the colors!:          https://en.wikibooks.org/wiki/LaTeX/Colors
%%   how to draw tables:       https://en.wikibooks.org/wiki/LaTeX/Tables
%%   code listing styles:      https://en.wikibooks.org/wiki/LaTeX/Source_Code_Listings
%%   \includegraphics          https://en.wikibooks.org/wiki/LaTeX/Importing_Graphics
%%   learn more about figures  https://en.wikibooks.org/wiki/LaTeX/Floats,_Figures_and_Captions
%%   automagic bibliography:   https://en.wikibooks.org/wiki/LaTeX/Bibliography_Management  (this one is kinda difficult the first time)
%%   REVTeX Guide:             http://www.physics.csbsju.edu/370/papers/Journal_Style_Manuals/auguide4-1.pdf
%%
%%   (this document is of class "revtex4-1", the REVTeX Guide explains how the class works)


%% CREATING THE .pdf FILE USING LINUX IN THE TERMINAL
%%
%% [terminal]$ pdflatex template.tex
%%
%% Run the command twice, always.
%% If you want to use \footnote, you need to run these commands (IN THIS SPECIFIC ORDER)
%%
%% [terminal]$ pdflatex template.tex
%% [terminal]$ bibtex template
%% [terminal]$ pdflatex template.tex
%% [terminal]$ pdflatex template.tex
%%
%% Don't ask me why, I don't know.

\usepackage{thmtools}
\DeclareMathOperator{\nullspace}{Nul}
\DeclareMathOperator{\collspace}{Col}
\DeclareMathOperator{\rref}{Rref}
%%\DeclareMathOperator{\dim}{Dim}

 % "meq": must be equal
\newcommand{\meq}{\overset{!}{=}}

\newcommand{\R}{\mathbb{R}}
\newcommand*\Heq{\ensuremath{\overset{\kern2pt L'H}{=}}}
\usepackage{bm}
\newcommand{\uveci}{{\bm{\hat{\textnormal{\bfseries\i}}}}}
\newcommand{\uvecj}{{\bm{\hat{\textnormal{\bfseries\j}}}}}
\DeclareRobustCommand{\uvec}[1]{{%
  \ifcsname uvec#1\endcsname
     \csname uvec#1\endcsname
   \else
    \bm{\hat{\mathbf{#1}}}%
   \fi
}}
\usepackage{siunitx}

\makeatletter
\newcommand*{\balancecolsandclearpage}{%
  \close@column@grid
  \cleardoublepage
  \twocolumngrid
}
\makeatother

\newcounter{subproject}
\renewcommand{\thesubproject}{\alph{subproject}}
\newenvironment{subproj}{
\begin{description}
	\item[\refstepcounter{subproject}(\thesubproject)]
}{\end{description}}


\begin{document}
\title{Numerical integration using linear algebra}   % self-explanatory
\author{Anders P. Åsbø}               % self-explanatory
\date{\today}
\noaffiliation                            % ignore this

\begin{abstract}
derp
\end{abstract}

\maketitle
\tableofcontents

\section{Introduction}\label{sec:1}
<<<<<<< HEAD
Numerical integration is an important cornerstone of computational physics, and as such it is important to understand its limits in terms of numerical precision and time spent computing. In this report I have looked at the specific case of numerically integrating a linear, second order differential equation as a system of linear equations, with the pretext of solving a one-dimensional variant of Poisson's equation.

\section{Formalism}\label{sec:2}
\subsection{Underlying theory}\label{subsec:21}
If I have a charge distrobution \(\rho(\vec{r})\), as a function of the position \(\vec{r}\), Possion's equation gives the electrostatic potential \(\Phi\)
$$
	\nabla^{2}=-4\pi\rho(\vec{r}).	
$$
If I then assume both \(\rho(\vec{r})\), and \(\Phi\) to bi spherically symetric, the equation can be simplified to
$$
	\frac{1}{r^{2}}\frac{d\Phi}{dr}r^{2}\frac{d\Phi}{dr}=-4\pi\rho(r),
$$
where \(r=|\vec{r}|\). Substituting \(\Phi(r)=\frac{1}{r}\phi(r)\) gives
$$
	\frac{d^{2}\phi}{dr^{2}}=-4\pi\rho(r),
$$
which, for simplicity, can be written as
$$
	-u''(x)=f(x),
$$
with \(u = \phi\), \(f=-4\pi\rho\), and \(x = r\).

Having arrived at a simple formulation of the initial problem, I can now discretize it using the second order differential approximation
$$
	-\frac{u(x+h)+u(x-h)-2u(x)}{h^{2}}=f(x).
$$
Since I will be opperating with a discreete set of variables \(x_{i}\in[0,1]\) where \(x_{i}=ih\) for \(i=1,2,3,...,N\), I can write the afformentioned approximation as
$$
	-\frac{u_{i+1}+u_{i-1}-2u_{i}}{h^{2}}=f_{i},
$$
where \(u_{i}=u(x_{i})\). The step-size is defined as \(h=\frac{x_{N}-x_{0}}{N}\), and I impose the Dirichlet boundary conditions \(u(0)=u(1)=0\).

\subsection{A general algorithm for square, tridiagonal matricies}\label{subsec:22}

The discretized approximation obtained in \hyperref[subsec:21]{II.B} can be rearranged to
$$
	-1u_{i-1}+2u_{i}-1u_{i+1}=h^{2}f_{i},
$$
which describes a tridiagonal, \(N\cross N\)-matrix
$$
	A =
	\begin{bmatrix}
	2 & -1 & 0 & \dots & 0 \\
	-1 & 2 & -1 & \ddots &\vdots \\
	0 & -1 & 2 & -1 & \vdots \\
	\vdots & \ddots & \ddots & \ddots & \vdots \\
	\vdots & \dots & \dots & -1 & 2 \\
	\end{bmatrix},
$$
and with the unkowns in a vector
$$
	\vec{u} =
	\begin{bmatrix}
		u_{1} \\
		\vdots \\
		u_{N} \\
	\end{bmatrix},
$$
and the values of \(h^{2}f_{i} = g_{i}\) in a vector
$$
	\vec{g} =
	\begin{bmatrix}
		g_{1} \\
		\vdots \\
		g_{N} \\
	\end{bmatrix},
$$
I can express the integration problem as a matrix equation
$$
	A\vec{u}=\vec{g}.
$$

For a general matrix
$$
	A =
	\begin{bmatrix}
	d_{1} & a_{1} & 0 & \dots & \dots & 0 \\
	b_{1} & d_{2} & a_{2} & \ddots & \ddots &\vdots \\
	0 & b_{2} & d_{3} & a_{3} & \ddots & \vdots \\
	\vdots & \ddots & \ddots & \ddots & \ddots & \vdots \\
	\vdots & \ddots & \ddots & \ddots & \ddots & a_{N-1} \\
	0 & \dots &\dots & \dots & b_{N-1} & d_{N} \\
	\end{bmatrix},
$$
A way to solve the matrix equation is by Gaussian elimination. I start by subtracting row 1 multiplied by \(\frac{b_{1}}{d_{1}}\) from row 2, and continue by subtracting row 2 multiplied by \(\frac{b_{2}}{d_{2}}\) from row 3. This continues all the way down such that a general algorithm can be written as
\begin{equation}
	d_{i}^{*}=d_{i}-\frac{b_{i-1}a_{i-1}}{d_{i-1}^{*}}, \label{eq:1}
\end{equation}
with the condition that \(d^{*}_{1}=d_{1}\). The same pattern aplies to \(\vec{g}\), such that
\begin{equation}
	g_{i}^{*} = g_{i}-\frac{b_{i-1}g_{i-1}}{d_{i-1}^{*}}, \label{eq:2}
\end{equation}
where \(g_{1}^{*} = g_{1}\), such that we get a new vector \(\vec{g}^{*}\) with the adjusted values. The two above algorithems constitutes the decomposition and forward substitution of the matrix, and gives \(u_{N}=\frac{g_{N}^{*}}{d_{N}^{*}}\). The remaining values of \(u\) can then be calculated recursivly as
\begin{equation}
		u_{i} = \frac{g_{i}^{*}-a_{i}u_{i+1}}{d_{i}^{*}}, \label{eq:3}
\end{equation}
with \(i = N-1,...,2,1\). This algorithm constitutes the backward substitution.

\section{Implementation}\label{sec:3}

\subsection{Implementing the general algorithm}\label{subsec:31}
To execute the numerical integration using \eqref{eq:1}, \eqref{eq:2}, and \eqref{eq:3}, I wrote the program \hyperref[A:1]{"project.py"(A.1)} which takes the number of step points \(N\), as well as a label for the data files, as input from the user, and initializes an array of linearly spaced values of \(x_{i}\in[0,1]\), as well as the step-size. Furthermore, it calls a custom module \hyperref[A:3]{"data\_generator.py"(A.3)}, which generates saves an array of \(f_{i}\) values to a textfile. \hyperref[A:3]{"data\_generator.py"(A.3)} also generates a seperate textfile containing an array of the analytical solution to the differential equation, since I am using the function
$$
	u(x) = 1-(1-e^{-10})x-e^{-10x},
$$
$$
	u'(x) = (1-e^{-10})-10e^{-10x},
$$
$$
	u''(x) = -100e^{-10x}=-f(x).
$$

Both the textfile containing \(f_{i}\), and the analytical solution are read by \hyperref[A:1]{"project.py"(A.1)}, which stores them as arrays. The program also initializes empty arrays for \(\vec{u}\), \(\vec{d}^{*}\), and \(\vec{g}^{*}\). The tridiagonal matrix is initialized as three arrays, one holding \(b_{i}\), one holding \(d_{i}\), and one holding \(a_{i}\). Thus I avoid filling memory with the zero-elements.

\hyperref[A:1]{"project.py"(A.1)} continues, by setting the boundary conditions, and looping through \eqref{eq:1}, and \eqref{eq:2} in one loop, and \eqref{eq:3} in a second loop. For the decomposition and forward substitution, the factor \(\frac{b_{i-1}}{d_{i-1}^{*}}\) is calculated before including it in \eqref{eq:1}, and \eqref{eq:2}, reducing the number of FLOPS from \(6(N-1)\) to \(5(N-1)\). Due to the boundary conditions, the backward substitution requires \(3(N-2)\) FLOPS, resulting in a total of \(8N-11\approx 8N\) FLOPS for sufficently large values of \(N\).

Finally, \hyperref[A:1]{"project.py"(A.1)} saves the numerial solution \(\vec{u}\) to a textfile, and plots it against the analytical solution.

\subsection{Implementing a specialized version of the algorithm}\label{subsec:32}
Because the specific tridiagonal matrix in the problem has all diagonal elements equal to \(2\), and all non-zero, off-diagonal elements equal to \(-1\), there is some optimization to be done, thus a specialized version of the integration program can be found in \hyperref[A:2]{"project\_specialized.py"(A.1)}.

Substituting in the known values, \eqref{eq:3} can be reduced to
$$
	g_{i}^{*} = g_{i} + \frac{g_{i-1}^{*}}{d_{i}^{*}},
$$
reducing the backward substitution to \(2(N-2)\) FLOPS.
Furthermore, the adjusted, diagonal elements can be expresed by an explicit formula
$$
	\frac{1}{d_{i}^{*}}=\frac{2i}{2(i+1)}.
$$
Utilizing NumPy's elementwise opperations, \hyperref[A:2]{"project\_specialized.py"(A.1)} precalculates all \(\frac{1}{d_{i}^{*}}\) in parallell, reducing the decomposition and forward substitution to \(2(N-1)\) FLOPS. This effectivly halves the number of FLOPS of the total algorithm to \(4N-6\approx 4N\), the arrays for the initial matrix elements were removed to as they are no longer needed.

\section{Analysis}\label{sec:4}

\subsection{Plotting the genneral algorithm}\label{subsec:41}

\subsection{Benchmarks of the general and specializeed algorithms}\label{subsec:42}

\subsection{Error analysis of the specialized algorithm}\label{subsec:43}

\subsection{Comparison with LU-decomposition}\label{subsec:43}

\section{Conclusion}\label{sec:5}

\appendix
\section{Program files} \label{A}
All code for this report was written in Python 3.6, and the complete set of files can be found at \url{https://github.com/FunkMarvel/CompPhys-Project-1}.
\subsection{project.py}\label{A:1}
=======

\section{Formalism}\label{sec:2}

\section{Implementation}\label{sec:3}

\section{Analysis}\label{sec:4}

\section{Conclusion}\label{sec:5}

\appendix
\section{Program files} \label{A:1}
All code for this report was written in Python 3.6, and the complete set of files can be found at \url{https://github.com/FunkMarvel/CompPhys-Project-1}.
\subsection{project.py}\label{A:11}
>>>>>>> 4998f6f102160946fb81fc765439eb4453bf1e45
\begin{lstlisting}
# Project 1 FYS3150, Anders P. Åsbø
# general tridiagonal matrix.
import data_generator as gen
import numpy as np
import os
import matplotlib.pyplot as plt
import timeit as time


def main():
    """Program solves matrix equation Au=f, using decomposition, forward
    substitution and backward substitution, for a tridiagonal, NxN matrix A."""
    init_data()  # initialising data

    # performing decomp. and forward and backward sub.:
    decomp_and_forward_and_backward_sub()

    save_sol()  # saving numerical solution in "data_files" directory.

    plot_solutions()  # plotting numerical solution vs analytical solution.

    plt.show()  # displaying plot.


def init_data():
    """Initialising data for program as global variables."""
    global dir, N, name, x, h, anal_sol, u, d, d_prime, a, b, g, g_prime
    dir = os.path.dirname(os.path.realpath(__file__))  # current directory.

    # defining number of rows and columns in matrix:
    N = int(eval(input("Specify number of data points N: ")))
    # defining common label for data files:
    name = input("Label of data-sets without file extension: ")

    x = np.linspace(0, 1, N)  # array of normalized positions.
    h = (x[0]-x[-1])/N  # defining step-siz.

    gen.generate_data(x, name)  # generating dataanal_name set.
    anal_sol = np.loadtxt("%s/data_files/anal_solution_for_%s.dat" %
                          (dir, name))

    u = np.empty(N)  # array for unkown values.
    d = np.full(N, 2)  # array for diagonal elements.
    d_prime = np.empty(N)  # array for diagonal after decom. and sub.
    a = np.full(N-1, -1)  # array for upper, off-center diagonal.
    b = np.full(N-1, -1)  # array for lower, off-center diagonal.
    # array for g in matrix eq. Au=g.
    f = np.loadtxt("%s/data_files/%s.dat" % (dir, name))
    g = f*h**2
    g_prime = np.empty(N)  # array for g after decomp. and sub.


def decomp_and_forward_and_backward_sub():
    """Function that performs the matrix decomposition and forward
    and backward substitution."""
    # setting boundary conditions:
    u[0], u[-1] = 0, 0
    d_prime[0] = d[0]
    g_prime[0] = g[0]

<<<<<<< HEAD
    start = time.default_timer()  # times algorithm
=======
    start = time.default_timer()
>>>>>>> 4998f6f102160946fb81fc765439eb4453bf1e45
    for i in range(1, len(u)):  # performing decomp. and forward sub.
        decomp_factor = b[i-1]/d_prime[i-1]
        d_prime[i] = d[i] - a[i-1]*decomp_factor
        g_prime[i] = g[i] - g_prime[i-1]*decomp_factor

    for i in reversed(range(1, len(u)-1)):  # performing backward sub.
        u[i] = (g_prime[i]-a[i]*u[i+1])/d_prime[i]
    end = time.default_timer()
    print("Time spent on loop %e" % (end-start))


def save_sol():
    """Function for saving numerical solution in data_files directory
    with prefix "solution"."""
    path = "%s/data_files/solution_%s.dat" % (dir, name)
    np.savetxt(path, u, fmt="%f")


def plot_solutions():
    """Function for plotting numerical vs analytical solutions."""
    x_prime = np.linspace(x[0], x[-1], len(anal_sol))

    plt.figure()
    plt.plot(x, u, label="Numerical solve")
    plt.plot(x_prime, anal_sol, label="Analytical solve")
    plt.title("Integrating with a %iX%i tridiagonal matrix" % (N, N))
    plt.xlabel(r"$x \in [0,1]$")
    plt.ylabel(r"$u(x)$")
    plt.legend()
    plt.grid()


if __name__ == '__main__':
    main()


# example run:
"""
$ python3 project.py
Specify number of data points N: 1000
Label of data-sets without file extension: num1000x1000
"""
# a plot is displayed, and the data is saved to the data_files directory.
<<<<<<< HEAD

\end{lstlisting}

\subsection{project\_specialized.py}\label{A:2}
=======
\end{lstlisting}

\subsection{project\_specialized.py}\label{A:12}
>>>>>>> 4998f6f102160946fb81fc765439eb4453bf1e45
\begin{lstlisting}
# Project 1 FYS3150, Anders P. Åsbø
import data_generator as gen
import numpy as np
import os
import matplotlib.pyplot as plt
import timeit as time


def main():
    """Program solves matrix equation Au=f, using decomposition, forward
    substitution and backward substitution, for a Toeplitz, NxN matrix A."""
    init_data()  # initialising data

    # performing decomp. and forward and backward sub.:
    decomp_and_forward_and_backward_sub()

    save_sol()  # saving numerical solution in "data_files" directory.

<<<<<<< HEAD
    # plot_solutions()  # plotting numerical solution vs analytical solution.

    # plt.show()  # displaying plot.
=======
    plot_solutions()  # plotting numerical solution vs analytical solution.

    plt.show()  # displaying plot.
>>>>>>> 4998f6f102160946fb81fc765439eb4453bf1e45


def init_data():
    """Initialising data for program as global variables."""
    global dir, N, name, x, h, anal_sol, u, d, d_prime, a, b, g, g_prime
    dir = os.path.dirname(os.path.realpath(__file__))  # current directory.

    # defining number of rows and columns in matrix:
    N = int(eval(input("Specify number of data points N: ")))
    # defining common label for data files:
    name = input("Label of data-sets without file extension: ")

    x = np.linspace(0, 1, N)  # array of normalized positions.
    h = (x[0]-x[-1])/N  # defining step-siz.

    gen.generate_data(x, name)  # generating dataanal_name set.
    anal_sol = np.loadtxt("%s/data_files/anal_solution_for_%s.dat" %
                          (dir, name))

    u = np.empty(N)  # array for unkown values.
    s = np.arange(1, N+1)
    d_prime = 2*(s)/(2*(s+1))  # pre-calculating the 1/d_prime factors.
    f = np.loadtxt("%s/data_files/%s.dat" % (dir, name))
    g = f*h**2
    g_prime = np.empty(N)  # array for g after decomp. and sub.


def decomp_and_forward_and_backward_sub():
    """Function that performs the matrix decomposition and forward
    and backward substitution."""
    # setting boundary conditions:
    u[0], u[-1] = 0, 0
    g_prime[0] = g[0]
    start = time.default_timer()
    for i in range(1, len(u)):  # performing decomp. and forward sub.
        g_prime[i] = g[i] + g_prime[i-1]*d_prime[i-1]

    for i in reversed(range(1, len(u)-1)):  # performing backward sub.
        u[i] = (g_prime[i] + u[i+1])*d_prime[i-1]

    end = time.default_timer()
<<<<<<< HEAD
    np.savetxt("looptime%i" % N, np.array([end-start]))
=======
    print("Time spent on loop %e" % (end-start))
>>>>>>> 4998f6f102160946fb81fc765439eb4453bf1e45


def save_sol():
    """Function for saving numerical solution in data_files directory
    with prefix "solution"."""
    path = "%s/data_files/solution_%s.dat" % (dir, name)
    np.savetxt(path, u, fmt="%f")


def plot_solutions():
    """Function for plotting numerical vs analytical solutions."""
    x_prime = np.linspace(x[0], x[-1], len(anal_sol))

    plt.figure()
    plt.plot(x, u, label="Numerical solve")
    plt.plot(x_prime, anal_sol, label="Analytical solve")
    plt.title("Integrating with a %iX%i tridiagonal matrix" % (N, N))
    plt.xlabel(r"$x \in [0,1]$")
    plt.ylabel(r"$u(x)$")
    plt.legend()
    plt.grid()


if __name__ == '__main__':
    main()


# example run:
"""
<<<<<<< HEAD
$ python3 project.py
=======
$ python3 project_specialized.py
>>>>>>> 4998f6f102160946fb81fc765439eb4453bf1e45
Specify number of data points N: 1000
Label of data-sets without file extension: opti1000x1000
"""
# a plot is displayed, and the data is saved to the data_files directory.

\end{lstlisting}

<<<<<<< HEAD
\subsection{data\_generator.py} \label{A:3}
\begin{lstlisting}
# create data set for numerical testing, Ander P. Åsbø
=======
\subsection{data\_generator.py}
\begin{lstlisting}
# create data set for numerical testing
>>>>>>> 4998f6f102160946fb81fc765439eb4453bf1e45
import numpy as np
import os

dir = os.path.dirname(os.path.realpath(__file__))


def main():
<<<<<<< HEAD
    """Generates a set of test-data, if run individually."""
    test_generate_data()
=======
    test_generate_data()
    test_generate_tridiagonal()
>>>>>>> 4998f6f102160946fb81fc765439eb4453bf1e45


def generate_data(x, name):
    """Function that generates a set of u''(x) data, as well as the
    corresponding, analytical u(x). The data is saved to text"""
    data = 100*np.exp(-10*x)
    path = "%s/data_files/%s.dat" % (dir, name)
    np.savetxt(path, data, fmt="%f")

<<<<<<< HEAD
    """
    # interpolated analytical solution used when plotting:
    x_prime = np.linspace(x[0], x[-1], 1000)
    analytical_solution = 1-(1-np.exp(-10))*x_prime-np.exp(-10*x_prime)
    """
    analytical_solution = 1-(1-np.exp(-10))*x-np.exp(-10*x)
=======
    x_prime = np.linspace(x[0], x[-1], 1000)
    analytical_solution = 1-(1-np.exp(-10))*x_prime-np.exp(-10*x_prime)
>>>>>>> 4998f6f102160946fb81fc765439eb4453bf1e45
    analytical_solution[0], analytical_solution[-1] = 0, 0
    anal_name = "%s/data_files/anal_solution_for_%s.dat" % (dir, name)
    np.savetxt(anal_name, analytical_solution, fmt="%f")


def generate_tridiagonal(N):
    """Function that generates a Nx3 array with each column corresponding to
    the non-zero elements in a tridiagonal matrix.
    "b" (mat_data[:,0]) is the lower diagonal,
    "d" (mat_data[:,1]) is the diagonal,
    and "a" (mat_data[:,2]) is the upper diagonal."""
    mat_data = np.random.randint(1, 100, size=(N, 3))
    np.savetxt("b-d-a_tridiagonal.dat", mat_data, fmt="%f")


def test_generate_data():
    """Generates test data if run as stand-alone program."""
    x = np.linspace(0, 1, 1001)
    test_name = "Test_data"
    generate_data(x, test_name)


<<<<<<< HEAD
if __name__ == '__main__':
    main()

# example run:
"""
$ python3 data_generator.py
"""
# the test-data files are sucessfully generated.

=======
def test_generate_tridiagonal():
    """Generates test data for tridiagonal."""
    generate_tridiagonal(100)
    print(np.loadtxt("b-d-a_tridiagonal.dat"))


if __name__ == '__main__':
    main()

>>>>>>> 4998f6f102160946fb81fc765439eb4453bf1e45
\end{lstlisting}

\bibliography{kilder}{}


\end{document}
